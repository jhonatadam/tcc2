\chapter{Considerações Finais}
\label{chap:consideracoes}

Neste trabalho, desenvolvemos um modelo de PI para o problema de alocação de professores e disciplinas da UFC-Quixadá. O modelo desenvolvido busca maximizar a preferência geral dos professores e minimizar o número de alunos que podem cursar pares de disciplinas com choque de horário.

Inicialmente fizemos um levantamento das restrições de alocação aplicadas atualmente na UFC-Quixadá. Modelamos essas restrições como um problema de PI, buscando tratar a maximização das preferências dos professores e a minimização dos choques de horários na função objetivo do modelo. Implementamos o modelo e realizamos testes convertendo as ofertas das disciplinas da UFC-Quixadá de 2016.2 como entrada para a implementação.

A partir dos testes realizados, ficou claro que a abordagem proposta por este trabalho não é efetiva para solucionar o problema de alocação da UFC-Quixadá. Mesmo obtendo uma solução viável para 100\% dos dados quando $\alpha = 0.5$, não há garantia de que as possíveis soluções viáveis encontradas pela implementação, sejam melhores do que as já produzida atualmente no Campus. Além disso, através do resultado para a instância com 90\% dos dados e com valor $\alpha = 0.5$, podemos ver que não há garantia nem de que encontremos uma solução viável antes que haja estouro de memória no processo de busca mesmo para instâncias menores.

Podemos concluir que é necessário buscar uma estratégia mais eficiente se quisermos realizar a alocação de professores e disciplinas para uma quantidade de dados de tamanho igual à que é alocada na UFC-Quixadá atualmente. 

Como trabalhos futuros, propomos os seguintes:
\begin{alineascomponto}
\item Observando os experimentos, percebemos que para instâncias menores (de 10\% e 20\%) o modelo se mostrou eficaz para encontrar uma solução ótima. Baseado nisso, uma possível abordagem para resolver o problema pode ser a aplicação de um modelo de PI para fazer a alocação das disciplinas de cada curso separadamente. Essa abordagem pode se mostrar efetiva, apesar de não garantir uma solução ótima para o problema de alocação completa do campus.
\item Comparar nosso método de alocação com outros métodos propostos na literatura. Existem diversas abordagens para problema de alocação que utilizam Inteligência Artificial. Talvez seja interessante a implementação de algum desses métodos realizando uma comparação com o nosso.
\item Dado que a alocação de professores na UFC-Quixadá já é feita de forma automatizada, algo interessante seria o desenvolvimento de um sistema com foco na facilidade de uso para usuários não especialistas no assunto. Isso eliminaria a necessidade de uma pessoa especialista realizando a tarefa de preparação dos dados, restando para essa pessoa apenas a tarefa de supervisionar todo o processo.
\end{alineascomponto}