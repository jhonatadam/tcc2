\chapter{Introdução}
\label{introducao}

Antes do início de um semestre letivo, as universidades realizam uma série de tarefas a fim de se preparar para as atividades de um novo semestre. Entre essas tarefas se encontra a alocação de professores em disciplinas, que lida com o problema de alocar professores em disciplinas e disciplinas nos horários de aula respeitando um conjunto de restrições. O problema se torna difícil de resolver quando lidamos com um grande volume de dados. Isso se dá por alguns motivos, que vão desde a quantidade de restrições envolvidas até às preferências concorrentes dos professores. Doravante, chamaremos esse problema de \textbf{PAPD (problema de alocação de professores em disciplinas)}.

No campus da Universidade Federal do Ceará em Quixadá (UFC-Quixadá), a direção do campus e os coordenadores de curso participam da definição da grade de horários acadêmicos. Nela está definida toda a alocação do campus, tanto de professores em disciplinas quanto de disciplinas na grade de horários. Atualmente as ofertas de disciplinas são definidas por cada curso. São definidos de forma manual a alocação dos professores a cada oferta, e de forma automatizada a alocação das disciplinas nos horários de aula. Essa alocação é feita respeitando um conjunto de restrições que definem como os professores e as disciplinas devem ser alocados.

Dentro do contexto de uma Universidade, conseguimos identificar três grupos de pessoas que são diretamente afetadas pelo PAPD, seja pela alocação gerada ou pelo método utilizado para fazer a alocação. São eles: os professores, os alunos e os responsáveis por gerar a alocação. Os professores têm suas preferências quanto às disciplinas que vão ministrar. Os alunos, por sua vez, são diretamente afetados por uma alocação que reduza suas opções de matrícula. Quanto aos responsáveis pela alocação, esses lidam com um volume de trabalho que cresce de forma exponencial em função da quantidade de dados envolvidos na alocação. Automatizar a resolução do PAPD, maximizando a preferência geral dos professores e as opções de matrícula dos alunos, é vantajoso para todas as pessoas afetadas pelo problema.

Uma larga variedade de problemas de alocação voltados à escolas de ensino médio e universidades são encontrados na literatura, bem como diversas abordagens são propostas para resolver o problema, dentre elas: Algoritmos Genéticos, Satisfação de Restrições, Fluxo em Redes e Programação Inteira, entre outras \cite{schaerf1999survey}. Neste trabalho, modelamos o PAPD como um problema de Programação Inteira. 

Um problema de Programação Inteira consiste em um conjunto de variáveis que compõem uma função objetivo e um conjunto de equações ou inequações lineares, que são as restrições do problema. Resolver um problema de Programação Inteira consiste em encontrar uma valor inteiro para as variáveis que satisfaça todas as restrições e minimize (ou maximize, dependendo da natureza do problema) o valor da função objetivo.

Este trabalho teve como objetivo geral produzir um modelo de Programação Inteira capaz de resolver o PAPD aplicado à UFC-Quixadá. Para tanto, identificamos as restrições que são utilizadas no processo de alocação. A partir das restrições levantadas, modelamos o PAPD da UFC-Quixadá como um problema de Programação Inteira e realizamos alguns testes implementando o modelo e verificando o tempo que ele leva para resolver o problema para instâncias de diferentes tamanhos.

No trabalho de \citeonline{naadriano}, é proposta uma forma de modelar o PAPD a partir dos conceitos da Teoria dos Jogos. Esse trabalho também leva em conta as restrições da alocação da UFC-Quixadá de forma que uma solução do modelo proposto no trabalho satisfaz essas restrições considerando a preferência por disciplinas dos professores. Assim como no trabalho de \citeonline{naadriano}, nosso trabalho leva em conta as restrições da alocação da UFC-Quixadá para produzir um modelo que maximiza a preferência por disciplinas dos professores. A diferença entre os dois trabalhos se encontra na abordagem utilizada para resolver o problema e na maximização das opções de matrícula que realizamos no nosso trabalho. 

\citeonline{lach2012curriculum} apresentam um modelo de Programação Inteira baseado no conjunto de restrições das instâncias da \textit{2nd International Timetabling Competition} \cite{itc}. Esse modelo resolve as instâncias em dois passos: primeiro  associa disciplinas com seus horários e depois associa esses horários às salas de aula. O que é importante ressaltar da abordagem é que ela trata o problema com dois tipos de restrições: \textit{Hard} e \textit{Soft}. Restrições \textit{Hard} devem ser respeitadas incondicionalmente; por outro lado, restrições \textit{Soft} devem ser satisfeitas tanto quanto possível. \citeonline{lach2012curriculum}, assim como no nosso trabalho, trabalham com Programação Inteira para resolver o problema de alocação em universidades. Os dois trabalhos se diferem na forma como resolvem problema, pois o nosso trabalho não divide em passos o processo de resolução. Além disso, os dois trabalhos lidam com um conjunto de restrições diferentes.

Esse trabalho está dividido em cinco capítulos. No Capítulo 2 é apresentada a fundamentação teórica. O Capítulo 3 apresenta o modelo de Programação Inteira desenvolvido para o PAPD da UFC-Quixadá. O Capítulo 4 apresenta a avaliação feita sobre o modelo desenvolvido. E por fim, no Capítulo 5, são feitas as considerações finais e as indicações de trabalhos futuros.

	