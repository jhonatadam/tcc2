%%
%%  O texto deste template é de autoria da professora Tania Saraiva de Melo Pinheiro (Universidade Federal do Ceará - Campus Quixadá)
%%
%%
%%
%%
%%

\chapter{Introdução}

Antes do início de um semestre letivo, as universidades realizam uma série de tarefas afim de se preparar para as atividades de um novo semestre. Entre essas tarefas se encontra a alocação de professores em disciplinas, que lida com o problema de alocar professores em disciplinas e disciplinas nos horários de aula respeitando um conjunto de restrições. O problema se torna difícil de resolver quando lidamos com um grande volume de dados. Isso se dá por alguns motivos, que vão desde a quantidade de restrições envolvidas até as preferências concorrentes dos professores. Doravante, chamaremos esse problema de \textbf{PAPD (problema de alocação de professores em disciplinas)}.

No campus da \textbf{Universidade Federal do Ceará em Quixadá (UFC-Quixadá)}, a direção do campus e os coordenadores de curso participam da definição da grade de horários acadêmicos. Nela está definida toda a alocação do campus, tanto de professores em disciplinas quanto de disciplinas na grade de horários.

A alocação de professores na UFC-Quixadá atualmente é feita de forma automatizada. (falar do que nosso modelo propõe de novo e qual a vantagem disso)

Dentro do contexto de uma Universidade, conseguimos identificar três grupos de pessoas que são diretamente afetadas pelo PAPD, seja pela alocação gerada ou pelo método utilizado para fazer a alocação. São eles: os professores, os alunos e os responsáveis por gerar a alocação. Os professores têm suas preferências quanto às disciplinas que vão ministrar. Eles também podem ter um conjunto de horários indesejáveis para dar aula, como por exemplo: a última aula da noite de terça e primeira aula da manhã de quarta. Os alunos por sua vez, são diretamente afetados por uma alocação que coloque em choque os horários de duas ou mais disciplinas que eles desejam cursar. Quanto aos responsáveis pela alocação, esses lidam com um volume de trabalho que cresce de forma exponencial em função da quantidade de dados envolvidos na alocação. 

No trabalho de Dodó (2011), é proposta uma forma de modelar o PAPD a partir dos conceitos da Teoria dos Jogos. Esse trabalho também leva em conta as restrições da alocação da UFC-Quixadá, de forma que uma solução do modelo proposto no trabalho satisfaz todas as restrições considerando a preferência por disciplinas dos professores. Lach e Lübbecke (2012) apresentam um modelo de Programação Linear Inteira baseado no conjunto de restrições das instâncias da \textit{2nd International Timetabling Competition} \cite{itc}. Esse modelo resolve as instâncias em dois passos: primeiro  associa disciplinas com seus horários e depois associa esses horários às salas de aula. Esse trabalho aborda o problema com dois tipos de restrições: \textit{Hard} e \textit{Soft}.

O restante deste artigo está divido da seguinte forma: na Seção 2, apresentamos uma breve definição de Programação Inteira; na Seção 3, apresentamos um modelo de Programação Inteira para o PAPD; por fim, na Seção 4 estão os trabalhos futuros desta pesquisa.

A primeira versão da introdução da monografia pode ser uma cópia da introdução do projeto de pesquisa correspondente. Esta versão inicial será revisada à medida que o texto avança. A principal revisão é quando se termina todo o trabalho. A introdução é a primeira parte a ser escrita, uma vez que guia todo o trabalho, e também a última ser revisada.
Enuncia-se o propósito geral do trabalho. A área é brevemente contextualizada, o que será aprofundado na seção trabalhos relacionados. Uma vez contextualizado percebe-se a relevância do estudo proposto bem como seu público alvo.
Ao final da introdução, ou perto do final, o objetivo geral e os específicos são enunciados na sequência do texto, sem usar marcadores. Ao longo de todo o trabalho, convém usar a palavra “objetivo” exclusivamente para se referir os objetivos do seu trabalho.
A capa e todo o texto devem ser digitados em fonte tamanho 12, e recomendamos Times New Roman. A alternativa é apenas Arial que, ao contrário do que sugere a norma, seu tamanho 11 é mais equivalente ao Times New Roman 12. Outros elementos poderão terão letra 10 pontos, como observado neste arquivo. Na dúvida, veja as normas. O texto deve ser justificado, exceto as referências no final do trabalho, que devem ser alinhadas a esquerda.
Todos os autores citados devem ter a referência incluída na lista de referências posicionada no final no trabalho.
Em trabalhos de graduação, encontram-se denominações das partes do trabalho como Capítulos ou como Seção, porque capítulo também é conhecido como seção primária. O autor pode optar pelo que preferir, desde que padronize a nomenclatura em todo o texto. Está previsto quebra de página entre as Seções primárias.
Tipicamente, a introdução é concluída apresentando cada Seção/Capítulo que a segue.
