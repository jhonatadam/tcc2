\chapter{Trabalhos Relacionados}

No cotidiano, um bom ponto de partida para se resolver um problema é procurar soluções já existentes para utilizá-las. Costumeiramente, as soluções que já existentes não se aplicam diretamente ao nosso caso, precisando ser adaptadas. Assim, antes de se começar a resolver questões de pesquisa, é preciso conhecer o que existe de mais atual no seu tema. 

Usando a abordagem de \citeonline{wazlawick2014metodologia} para explicar a necessidade de se conhecer a área de estudo, cabe lembrar que antes de se construir uma nova ponte é importante conhecer os tipos de pontes que já existem; é preciso conhecer qual a atualidade do assunto estudado. Do contrário, pode estar construindo uma catapulta achando que se trata da melhor forma de atravessar um rio!

Para conhecer a atualidade do tema de estudo proposto, o ideal seria fazer o vasto levantamento do que se tem estudado ou praticado sobre o tema. Entretanto, em cursos de graduação em geral não há tempo para tanto, a menos que o objeto do estudo seja justamente o levantamento do estado da arte. 
Na impossibilidade de realiza-lo, deve-se pelo menos fazer um levantamento por amostragem. Tal amostra consiste de um conjunto de trabalhos relacionados: uma boa seleção de textos encontrados em periódicos e eventos relevantes para a área estudada.  O levantamento é facilitado quando se encontram materiais denominados surveys (levantamentos), podendo ser compilações de:

\begin{alineascomponto}
    \item \textbf{Estado-da-arte}: artigos que apresentem conceitos mais recentes, estabelecidos na literatura científica;
    \item \textbf{Estado-da-prática}: semelhante ao anterior, mas com foco no que está estabelecido atualmente como status quo da prática profissional.
\end{alineascomponto}

Na escrita desta seção, deve-se evitar usar a palavra “trabalho” para se referir tanto à própria pesquisa quanto à de outro autor, sugere-se:

\begin{alineascomponto}
\item em vez de “o trabalho de Bittar (2001) prevê que ...”,
\item utilizar-se simplesmente “Bittar (20010 prevê que....”.
\end{alineascomponto}

Apenas a partir do momento em que se conhece o estado da arte (ou prática), seja plenamente ou por uma amostra de trabalhos relacionados, é que o pesquisador está pronto para adequadamente identificar e possíveis pesquisas a serem realizadas. O anúncio de seus objetivos, portanto, ocorre após tecer considerações sobre o estado do conhecimento ou prática em sua área de estudo.

\section{Quando parece ser cabível o inverso}

Em algumas situações, o pesquisador tende a apresentar primeiro seu objeto de estudo e apenas depois o estado da arte. Observa-se que tais situações ocorrem quando o problema de pesquisa é extraído do cotidiano do pesquisador. Por exemplo: “não estou conseguindo bom resultado com determinado processo de trabalho, e vou pesquisar como melhorá-lo”.

Nesta situação, há uma tendência a primeiro se definir objetivo e só depois fazer um levantamento do estado da arte/prática. Mas qual seria então o propósito do tal levantamento? Buscar soluções semelhantes que auxiliem na elaboração da solução buscada? Se assim o for, então tais trabalhos relacionados não estariam também contribuindo para um refinamento na definição dos objetivos da pesquisa?