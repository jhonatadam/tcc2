Universities often face the problem of professors and courses scheduling. In this work, we address this problem by maximizing the general preference of professors for courses and minimizing time overlap on courses in which more students may enroll. The goal of this work is to develop a Integer Programming Model able to solving the problem of professors and courses scheduling to be applied on Quixadá Campus of the Federal University of Ceará. The construction of the Integer Programming model is presented step by step based on the Campus scheduling constraints. Using offered courses of the Campus in the semester of 2016.2, we perform experiments from the implementation of the model. The experiments showed that the model is not effective to the problem of professors and courses scheduling for instances that contain 100\% of the data to be allocated currently on Campus.

\palavraschave{Professors Scheduling. Courses Scheduling. Integer Programming. UFC-Quixadá.}
