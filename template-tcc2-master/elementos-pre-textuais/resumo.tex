As universidades lidam frequentemente com o problema de alocação de professores e disciplinas. Neste trabalho abordamos esse problema buscando maximizar a preferencia geral dos porfessores por disciplinas e minimizar o choque de horário entre disciplinas em que mais alunos possam se matricular. O objetivo desse trabalho é desenvolver um modelo de Programação Inteira capaz de resolver o problema de alocação de professores e disciplinas aplicado ao Campus da Universidade Federal do Ceará em Quixadá. É apresentado o passo a passo da construção do modelo de Programação Inteira baseado nas restrições de alocação praticadas no Campus. A partir da implementação do modelo foram realizados experimentos para instâncias produzidas com as ofertas de disciplinas do Campus do semestre de 2016.2. Os experimentos mostraram que o modelo não é efetivo na alocação de professores e disciplinas para instâncias com 100\% dos dados a serem alocados atualmente no Campus.

% Separe as palavras-chave por ponto
\palavraschave{Alocação de Professores. Alocação de Disciplinas. Programação Inteira. UFC-Quixadá.}

% Um modelo de Programação Inteira consiste em um conjunto de variáveis que compõem uma função objetivo e um conjunto de equações ou inequações lineares que representam as restrições do modelo. Resolver um problema de Programação Inteira, se resume a encontrar uma valor inteiro para as variáveis, que satisfaça todas as restrições e minimizes o custo da função objetivo.